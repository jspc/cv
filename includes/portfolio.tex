\section{Projects}
\begin{tabular*}{\textwidth}{@{\extracolsep{\fill}}ll}
  \entry
  {2022 - 2023}
  {Kaluza Gender Equality Community}
  {Kaluza - a Green Energy PaaS}
  {The Kaluza GEC exist within the wider Kaluza DE\&I space to provide advocacy for people across the gender spectrum at Kaluza. Our accomplishments include getting parity between Paternity and Maternity leave, providing environmentally friendly period products, and launching Ask Ada - an anonymous service for colleagues to ask DE\&I related questions and seek advice on matters, written in a mixture of golang (for streams processing), python (for sentiment analysis), and node.js (for developer tooling)}
\end{tabular*}

\section{Portfolio}

\begin{tabular*}{\textwidth}{@{\extracolsep{\fill}}ll}
  \entry
  {2023}
  {Ask Ada}
  {https://github.com/gender-equality-community}
  {Ask Ada is a system which bridges a whatsapp chat bot with a slack based backend, passing messages over Redis Streams, and deployed to Kubernetes with a custom operator. It exists to provide anonymous answers and advice on DE\&I questions without either party knowing who the other is}

  \entry
  {2021}
  {SeoulOS}
  {https://github.com/jspc/seoulos}
  {SeoulOS is a micro-kernel research project written in rust. It aims to be a proof-of-concept for running standard workloads on clusters of commodity hardware}

  \entry
  {2020-}
  {Vinyl Linux}
  {https://github.com/vinyl-linux}
  {Vinyl Linux is a Linux distribution centred around \texttt{vin}, a super-fast package manager written as a client/server pair in golang, and \texttt{vinyl linux-utils}, a \texttt{busybox}-alike set of standard tools, also written in golang. Vinyl Linux can parse and install \texttt{apk} packages from Alpine Linux}

  \entry
  {2016, 2019-2023}
  {This CV}
  {https://github.com/jspc/cv}
  {My CV is written in \LaTeX, and is built and deployed via \href{https://github.com/jspc/cv/actions}{Github Actions} and published to https://jspc.pw/cv.html}

  \entry
  {2018}
  {go-lo}
  {https://github.com/go-lo/go-lo}
  {A distributed load-testing platform, written in golang. \texttt{go-lo} is designed to run consistent, low-overhead load-tests. These load-tests are self contained golang apps which expose a control plane over gRPC and output results to \texttt{STDOUT} which are then forwarded into the TICK stack.}

\end{tabular*}
